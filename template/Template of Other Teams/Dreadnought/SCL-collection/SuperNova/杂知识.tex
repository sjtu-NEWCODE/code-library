\subsection*{牛顿迭代}
x1=x0-func(x0)/func1(x0);进行牛顿迭代计算\\
我们要求 f(x)=0 的解。func(x)为原方程,func1 为原方程的导数方程
\subsection*{图同构 Hash}
	\begin{displaymath}
		F_t(i) = (F_{t - 1}(i) * A + \sum_{i \to j} (F_{t - 1}(j) * B) + \sum_{j \to i} (F_{t - 1}(j) * C) + D * (i == a)) \mod P
	\end{displaymath}
	枚举点a, 迭代K次后求得的$F_k(a)$就是a点所对应的hash值。\\
	其中 K、A、B、C、D、P 为 hash 参数,可自选。

\subsection*{圆上有整点的充要条件}
设正整数 n 的质因数分解为 $n = \Pi p_i^{a_i}$,则 $x^2+y^2=n$ 有整数解的充要条件是 n 中不存在形
如 $p_i \mod 4 = 3$ 且指数 $a_i$ 为奇数的质因数 $p_i$

\subsection*{Pick 定理}
	简单多边形,不自交。(严格在多边形内部的整点数*2 + 在边上的整点数– 2)/2 = 面积

\subsection*{图定理}
	定理 1:最小覆盖数 = 最大匹配数\\
	定理 2:最大独立集 S 与 最小覆盖集 T 互补。\\
	算法:\\
	1. 做最大匹配,没有匹配的空闲点$\in S$\\
	2. 如果 $u \in S$ 那么 u 的临点必然属于 T\\
	3. 如果一对匹配的点中有一个属于 T 那么另外一个属于 S\\
	4. 还不能确定的,把左子图的放入 S,右子图放入 T\\
	算法结束\\

\subsection*{梅森素数}
	p 是素数且 $2^p-1$ 的是素数,n 不超过 258 的全部梅森素数终于确定!是:\\
	n=2,3,5,7,13,17,19,31,61,89,107,127

\subsection*{上下界网络流}
	有上下界网络流,求可行流部分,增广的流量不是实际流量。若要求实际流量应该强算一遍源点出去的流量。\\
	求最小下届网络流:\\
	方法一:加 t-s 的无穷大流,求可行流,然后把边反向后(减去下届网络流),在残留网络中从汇到源做最大流。\\
	方法二:在求可行流的时候,不加从汇到源的无穷大边,得到最大流 X, 加上从汇到源无穷大边后,再求最大流得到 Y。那么 Y 即是答案最小下界网络流。\\
	原因:感觉上是在第一遍已经把内部都消耗光了,第二遍是必须的流量。

\subsection*{平面图定理}
	平面图一定存在一个度小于等于 5 的点,且可以四染色\\
	( 欧拉公式 ) 设 G 是连通的平面图,n,m,r分别是其顶点数、边数和面数,n-m+r=2\\
	极大平面图 $m\leq 3n-6$

\subsection*{Fibonacci相关结论}
	gcd(F[n],F[m])=F[gcd(n,m)]\\
	Fibonacci 质数(和前面所有的 Fibonacci 数互质), 下标为质数或4\\
	定理:如果 a 是 b 的倍数,那么 F[a] 是 F[b] 的倍数。\\

\subsection*{二次剩余}
	p 为奇素数,若(a,p)=1, a 为 p 的二次剩余必要充分条件为 $a^{(p-1)/2} \mod p=1$.(否则为$p-1$)\\
	p 为奇素数, $x^b=a(\mod p)$,a 为 p 的 b 次剩余的必要充分条件为若 $a^{(p-1)/ (p-1, b)} \mod p=1$.
