\subsection{巴什博奕}
	\begin{enumerate}\setlength{\itemsep}{-\itemsep}
		\item 
			只有一堆n个物品,两个人轮流从这堆物品中取物,规定每次至少取一个,最多取m个。最后取光者得胜。
		\item
			显然,如果$n=m+1$,那么由于一次最多只能取$m$个,所以,无论先取者拿走多少个,
			后取者都能够一次拿走剩余的物品,后者取胜。因此我们发现了如何取胜的法则:如果
			$n=(m+1)r+s$,(r为任意自然数,$s \leq m$),那么先取者要拿走$s$个物品,
			如果后取者拿走$k(k \leq m)$个,那么先取者再拿走$m+1-k$个,结果剩下$(m+1)(r-1)$
			个,以后保持这样的取法,那么先取者肯定获胜。总之,要保持给对手留下$(m+1)$的倍数,
			就能最后获胜。
	\end{enumerate}
\subsection{威佐夫博弈}
	\begin{enumerate}\setlength{\itemsep}{-\itemsep}
		\item 
			有两堆各若干个物品,两个人轮流从某一堆或同时从两堆中取同样多的物品,规定每次至少取
			一个,多者不限,最后取光者得胜。
		\item
			判断一个局势$(a, b)$为奇异局势(必败态)的方法:
			$$a_k =[k (1+\sqrt{5})/2],b_k= a_k + k$$
	\end{enumerate}
\subsection{阶梯博奕}
	\begin{enumerate}\setlength{\itemsep}{-\itemsep}
		\item
			博弈在一列阶梯上进行,每个阶梯上放着自然数个点,两个人进行阶梯博弈,
			每一步则是将一个阶梯上的若干个点(至少一个)移到前面去,最后没有点
			可以移动的人输。
		\item
			解决方法:把所有奇数阶梯看成N堆石子,做NIM。(把石子从奇数堆移动到偶数
			堆可以理解为拿走石子,就相当于几个奇数堆的石子在做Nim)
	\end{enumerate}
\subsection{图上删边游戏}
	\subsubsection{链的删边游戏}
		\begin{enumerate}\setlength{\itemsep}{-\itemsep}
			\item
				游戏规则:对于一条链,其中一个端点是根,两人轮流删边,脱离根的部分也算被删去,最后没边可删的人输。
			\item
				做法:$sg[i] = n - dist(i) - 1$(其中$n$表示总点数,$dist(i)$表示离根的距离)
		\end{enumerate}
	\subsubsection{树的删边游戏}
		\begin{enumerate}\setlength{\itemsep}{-\itemsep}
			\item
				游戏规则:对于一棵有根树,两人轮流删边,脱离根的部分也算被删去,没边可删的人输。
			\item
				做法:叶子结点的$sg=0$,其他节点的$sg$等于儿子结点的$sg+1$的异或和。
		\end{enumerate}
	\subsubsection{局部连通图的删边游戏}
		\begin{enumerate}\setlength{\itemsep}{-\itemsep}
			\item
				游戏规则:在一个局部连通图上,两人轮流删边,脱离根的部分也算被删去,没边可删的人输。
				局部连通图的构图规则是,在一棵基础树上加边得到,所有形成的环保证不共用边,且只与基础树有一个公共点。
			\item
				做法:去掉所有的偶环,将所有的奇环变为长度为1的链,然后做树的删边游戏。
		\end{enumerate}
