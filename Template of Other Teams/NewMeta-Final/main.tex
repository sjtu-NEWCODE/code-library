\documentclass[titlepage, a4paper,10pt]{article}
\usepackage{listings, color, fontspec, minted, setspace, titlesec, fancyhdr, dingbat, mdframed, multicol}
\usepackage{graphicx, amssymb, amsmath, textcomp, booktabs}
\usepackage[Chinese]{ucharclasses}
\usepackage[left=1.5cm, right=0.7cm, top=1.7cm, bottom=0.0cm]{geometry}
\usepackage{pdfpages}
\usepackage{tocloft}

%configure the top corners
\pagestyle{fancy}
\setlength{\headsep}{0.1cm}
\rhead{Page \thepage}
\lhead{上海交通大学 Shanghai Jiao Tong University}

%configure space between the two columns
\setlength{\columnsep}{30pt}

%configure fonts
\setmonofont{Isotype}[Scale=0.8]
\newfontfamily\substitutefont{SimHei}[Scale=0.8]
\setTransitionsForChinese{\begingroup\substitutefont}{\endgroup}

%configure minted to display codes
\definecolor{Gray}{rgb}{0.9,0.9,0.9}

%remove leading numbers in table of contents
%\setcounter{secnumdepth}{0}

%configure section style of table of content
\renewcommand\cftsecfont{\Large}

%configure section style
\titleformat{\section}
	{\huge}			% The style of the section title
	{\thesection.}				% a prefix
	{4pt}						% How much space exists between the prefix and the title
	{}					% How the section is represented
% \titleformat{\section}{\huge}{}{0pt}{}
\titlespacing{\section}{0pt}{0pt}{0pt}

%enable section to start new page automatically
%\let\stdsection\section
%\renewcommand\section{\penalty-100\vfilneg\stdsection}

%\renewcommand\theFancyVerbLine{\arabic{FancyVerbLine}}
\renewcommand{\theFancyVerbLine}{\sffamily \textcolor[rgb]{0.5,0.5,0.5}{\scriptsize {\arabic{FancyVerbLine}}}}

\setminted[cpp]{
	style=xcode,
	mathescape,
	linenos,
	autogobble,
	baselinestretch=0.9,
	tabsize=2,
	fontsize=\normalsize,
	%bgcolor=Gray,
	frame=single,
	framesep=1mm,
	framerule=0.3pt,
	numbersep=1mm,
	breaklines=true,
	breaksymbolsepleft=2pt,
	%breaksymbolleft=\raisebox{0.8ex}{ \small\reflectbox{\carriagereturn}}, %not moe!
	%breaksymbolright=\small\carriagereturn,
	breakbytoken=false,
}
\setminted[java]{
	style=xcode,
	mathescape,
	linenos,
	autogobble,
	baselinestretch=1.0,
	tabsize=2,
	%bgcolor=Gray,
	frame=single,
	framesep=1mm,
	framerule=0.3pt,
	numbersep=1mm,
	breaklines=true,
	breaksymbolsepleft=2pt,
	%breaksymbolleft=\raisebox{0.8ex}{ \small\reflectbox{\carriagereturn}}, %not moe!
	%breaksymbolright=\small\carriagereturn,
	breakbytoken=false,
}
\setminted[text]{
	style=xcode,
	mathescape,
	linenos,
	autogobble,
	baselinestretch=1.0,
	tabsize=2,
	%bgcolor=Gray,
	frame=single,
	framesep=1mm,
	framerule=0.3pt,
	numbersep=1mm,
	breaklines=true,
	breaksymbolsepleft=2pt,
	%breaksymbolleft=\raisebox{0.8ex}{ \small\reflectbox{\carriagereturn}}, %not moe!
	%breaksymbolright=\small\carriagereturn,
	breakbytoken=false,
}

%configure titles
%\title{\LARGE{New Meta} \\[2ex] \Large{Standard Code Library} }
%\date{\today}

%THE SCL BEGINS
\begin{document}
	\include{}
\end{document}
\begin{document}

\begin{multicols*}{2}

\tableofcontents
\end{multicols*}

\begin{multicols}{2}

\newpage
\begin{spacing}{0.8}

\section{Geometry}

\subsection{三维几何}
\inputminted{cpp}{1 Geometry/1 3DGeo.cpp}

\subsection{三维凸包}
\inputminted{cpp}{1 Geometry/2 3DConvex.cpp}

\subsection{阿波罗尼茨圆}
\inputminted{cpp}{1 Geometry/3 阿波罗尼茨圆.cpp}

\subsection{最小覆盖球}
\inputminted{cpp}{1 Geometry/4 最小覆盖球.cpp}

\subsection{三角形与圆交}
\inputminted{cpp}{1 Geometry/5 areaCT.cpp}

\subsection{圆并}
\inputminted{cpp}{1 Geometry/6 CircleArea.cpp}

\subsection{Delaunay 三角剖分}
\inputminted{cpp}{1 Geometry/7 DelaunayTriangulation.cpp}

\subsection{二维几何}
\inputminted{cpp}{1 Geometry/8 Geo2D_simple.cpp}

\subsection{整数半平面交}
\inputminted{cpp}{1 Geometry/9 HPI_integer.cpp}

\subsection{凸包闵可夫斯基和}
\inputminted{cpp}{1 Geometry/10 mink.cpp}

\subsection{三角形}
\inputminted{cpp}{1 Geometry/11 triangle.cpp}

\subsection{经纬度求球面最短距离}
\inputminted{cpp}{1 Geometry/12 经纬度求球面最短距离.cpp}

\subsection{长方体表面两点最短距离}
\inputminted{cpp}{1 Geometry/13 长方体表面两点最短距离.cpp}

\subsection{点到凸包切线}
\inputminted{cpp}{1 Geometry/14 点到凸包切线.cpp}

\subsection{直线与凸包的交点}
\inputminted{cpp}{1 Geometry/15 直线与凸包的交点.cpp}


\section{Graph}

% \subsection{Conclusion}
% \inputminted{cpp}{2 Graph/0 Conclusion.cpp}

\subsection{无向图最小割}
\inputminted{cpp}{2 Graph/1 无向图最小割.cpp}

\subsection{Blossom Algorithm}
\inputminted{cpp}{2 Graph/2 Blossom_Algorithm.cpp}

\subsection{仙人掌}
\inputminted{cpp}{2 Graph/3 cactus.cpp}

\subsection{最小树形图}
\inputminted{cpp}{2 Graph/4 ChuLiu.cpp}

\subsection{Dominator Tree}
\inputminted{cpp}{2 Graph/5 dominator_tree.cpp}

\subsection{离线动态最小生成树}
\inputminted{cpp}{2 Graph/6 DynamicMST.cpp}

\subsection{GH Tree}
\inputminted{cpp}{2 Graph/7 GHTree.cpp}

\subsection{Hopcroft mathcing}
\inputminted{cpp}{2 Graph/8 Hopcroft.cpp}

\subsection{KM}
\inputminted{cpp}{2 Graph/9 KM.cpp}

\subsection{Maximum Clique}
\inputminted{cpp}{2 Graph/10 MaximumClique.cpp}

\subsection{原始对偶费用流}
\inputminted{cpp}{2 Graph/11 MCMF_dij.cpp}

\subsection{完美消除序列}
\inputminted{cpp}{2 Graph/12 PEO.cpp}

\subsection{Tarjan}
\inputminted{cpp}{2 Graph/13 Tarjan.cpp}

\subsection{ZKW 费用流}
\inputminted{cpp}{2 Graph/14 ZKW费用流.cpp}

\section{String}

\subsection{Exkmp}
\inputminted{cpp}{3 String/1 扩展KMP.cpp}

\subsection{Lyndon Word Decomposition}
\inputminted{cpp}{3 String/2 Lyndon Word Decomposition.cpp}

\subsection{Manacher}
\inputminted{cpp}{3 String/3 Manacher.cpp}

\subsection{Minimum Representation}
\inputminted{cpp}{3 String/4 Minimum Representation.cpp}

\subsection{Palindromic Automaton}
\inputminted{cpp}{3 String/5 Palindromic Automaton.cpp}

\subsection{Suffix Array}
\inputminted{cpp}{3 String/6 Suffix Array.cpp}

\subsection{Suffix Automaton}
\inputminted{cpp}{3 String/7 Suffix Automaton.cpp}

\subsection{AC 自动机}
\inputminted{cpp}{3 String/8 AC自动机.cpp}

\subsection{子串最长公共子序列}
\inputminted{cpp}{3 String/9 SubStringLCS.cpp}

\section{Tree}

\subsection{树点分治-斜率优化}
\inputminted{cpp}{4 Tree/1 树点分治(斜率优化).cpp}

\subsection{树链剖分}
\inputminted{cpp}{4 Tree/2 树链剖分.cpp}

\subsection{虚树}
\inputminted{cpp}{4 Tree/3 虚树.cpp}

\subsection{有根树同构}
\inputminted{cpp}{4 Tree/4 有根树同构.cpp}

\section{Math}

\subsection{Conclusions}
\inputminted{cpp}{5 Math/0 Conclusion.cpp}

\subsection{积性函数线性求法}
\inputminted{cpp}{5 Math/1 积性函数线性求法.cpp}

\subsection{平方剩余}
\inputminted{cpp}{5 Math/2 平方剩余.cpp}

\subsection{线性同余不等式}
\inputminted{cpp}{5 Math/3 线性同余不等式.cpp}

\subsection{Schreier Sims}
\inputminted{cpp}{5 Math/4 SchreierSims.cpp}

\subsection{CRT}
\inputminted{cpp}{5 Math/5 CRT.cpp}

\subsection{Factorial Mod}
\inputminted{cpp}{5 Math/6 FactorialMod.cpp}

\subsection{Miller Rabin and Pollard Rho}
\inputminted{cpp}{5 Math/7 Miller Rabin And Pollard Rho.cpp}

\subsection{Pell 方程}
\inputminted{cpp}{5 Math/8 Pell方程.cpp}

\subsection{Simplex}
\inputminted{cpp}{5 Math/9 Simplex.cpp}

\subsection{Simpson}
\inputminted{cpp}{5 Math/10 Simpson.cpp}

\subsection{FFT}
\inputminted{cpp}{5 Math/11 FFT.cpp}
\begin{footnotesize}
	\begin{align}
	A(x)B(x)&\equiv 1\pmod{x^n}\\
	(A(x)B(x)-1)^2&\equiv 0\pmod{x^{2n}}\\
	A(x)(2B(x)-B(x)^2A(x))&\equiv 1\pmod{x^{2n}}
	\end{align}
	\begin{align}
	B(x)&=\ln{A(x)}\\
	B'(x)&=\frac{A'(x)}{A(x)}
	\end{align}
	\begin{align}
	f(x)&=\exp{A(x)}\\
	g(f(x))&=\ln{f(x)}-A(x)=0\\
	f(x)&\equiv f_0(x)\pmod{x^n}\\
	f(x)&\equiv f_0(x)(1-\ln{f_0(x)}+A(x))\pmod{x^{2n}}
	\end{align}
\end{footnotesize}

\subsection{解一元三次方程}
\inputminted{cpp}{5 Math/12 解一元三次方程.cpp}

\subsection{线性递推}
\inputminted{cpp}{5 Math/13 线性递推.cpp}

\subsection{黑盒子代数}
\inputminted{cpp}{5 Math/14 黑盒子代数.cpp}

\section{Data Structure}

% \subsection{Conclusion}
% \inputminted{cpp}{6 Data Structure/0 Conclusion.cpp}

\subsection{可持久化左偏树-K短路}
\inputminted{cpp}{6 Data Structure/1 可持久化左偏树(K短路).cpp}

\subsection{左偏树}
\inputminted{cpp}{6 Data Structure/2 左偏树.cpp}

\subsection{KD 树}
\inputminted{cpp}{6 Data Structure/3 KD树.cpp}

\subsection{LCT}
\inputminted{cpp}{6 Data Structure/4 LCT.cpp}

\subsection{Merge-Split Treap}
\inputminted{cpp}{6 Data Structure/5 Merge-Split Treap.cpp}

\subsection{Splay}
\inputminted{cpp}{6 Data Structure/6 Splay.cpp}

\section{Miscellany}

\subsection{日期公式}
\inputminted{cpp}{7 Miscellany/1 日期公式.cpp}

\subsection{DLX - 主代码手}
\inputminted{cpp}{7 Miscellany/2 DancingLinks.cpp}

\subsection{直线下格点统计}
\inputminted{cpp}{7 Miscellany/3 直线下格点统计.cpp}

\section{Others}

\subsection{Java Template}
\inputminted{java}{Others/Main.java}

\subsection{Formulas}
\def \bangle{ \atopwithdelims \langle \rangle}
\begin{small}
\subsubsection{Arithmetic Function}
\[ \sigma_k(n) = \sum_{d|n}d^k = \prod_{i=1}^{\omega(n)}\frac{p_i^{(a_i+1)k}-1}
{p_i^k-1} \]
\[ J_k(n) = n^k\prod_{p|n}(1-\frac{1}{p^k}) \]
$J_k(n)$ is the number of $k$-tuples of positive integers all less than or equal to n that form a coprime $(k + 1)$-tuple together with $n$.
\[ \sum_{\delta|n}J_k(\delta) = n^k \]
\[ \sum_{\delta|n}\delta^sJ_r(\delta)J_s(\frac{n}{\delta}) = J_{r+s}(n) \]
\begin{align*}
\sum_{\delta|n}\varphi(\delta)d(\frac{n}{\delta}) = \sigma(n),&\ \sum_{\delta|n}\left| \mu(\delta) \right| = 2^{\omega(n)} \\
\sum_{\delta|n}2^{\omega(\delta)} = d(n^2),&\ \sum_{\delta|n}d(\delta^2) = d^2(n) \\
\sum_{\delta|n}d(\frac{n}{\delta})2^{\omega(\delta)} = d^2(n),&\ \sum_{\delta|n}\frac{\mu(\delta)}{\delta} = \frac{\varphi(n)}{n} \\
\sum_{\delta|n}\frac{\mu(\delta)}{\varphi(\delta)} = d(n),&\ \sum_{\delta|n}\frac{\mu^2(\delta)}{\varphi(\delta)} = \frac{n}{\varphi(n)} \\
\end{align*}
\[ n|\varphi(a^n-1) \]
\[ \sum_{\substack{1 \leq k \leq n \\ \gcd(k, n) = 1}}f(\gcd(k-1, n)) = \varphi(n)
\sum_{d|n}\frac{(\mu*f)(d)}{\varphi(d)} \]
\[ \varphi(\mathrm{lcm}(m, n))\varphi(\gcd(m,n)) = \varphi(m)\varphi(n) \]
\[ \sum_{\delta|n}d^3(\delta) = (\sum_{\delta|n}d(\delta))^2 \]
\[ d(uv) = \sum_{\delta|\gcd(u, v)}\mu(\delta)d(\frac{u}{\delta})d(\frac{v}{\delta}) \]
\[ \sigma_k(u)\sigma_k(v) = \sum_{\delta|\gcd(u, v)}\delta^k\sigma_k(\frac{uv}{\delta^2}) \]
\[ \mu(n) = \sum_{k=1}^n[\gcd(k, n)=1]\cos{2\pi \frac{k}{n}} \]
\[ \varphi(n) = \sum_{k=1}^n[\gcd(k, n)=1] = \sum_{k=1}^n\gcd(k, n)\cos{2\pi \frac{k}{n}} \]
\[
\left\{
\begin{aligned}
&S(n) = \sum_{k=1}^n(f \ast g)(k) \\
&\sum_{k=1}^nS(\lfloor {n \over k} \rfloor) = \sum_{i=1}^nf(i)\sum_{j=1}^{\lfloor{n \over i}\rfloor}(g \ast 1)(j)
\end{aligned}
\right.
\]
\[
\left\{
\begin{aligned}
&S(n) = \sum_{k=1}^n(f \cdot g)(k), g \text{ completely multiplicative} \\
&\sum_{k=1}^nS(\lfloor {n \over k} \rfloor)g(k) = \sum_{k=1}^n(f \ast 1)(k)g(k)
\end{aligned}
\right.
\]
\subsubsection{Binomial Coefficients}
\[ {n \choose k} = (-1)^k{k-n-1 \choose k} \]
\[ \sum_{k \leq n}{r+k \choose k} = {r+n+1 \choose n} \]
\[ \sum_{k=0}^n{k \choose m} = {n+1 \choose m+1} \]
\[ \sqrt{1+z} = 1 + \sum_{k=1}^{\infty}\frac{(-1)^{k-1}}{k\times2^{2k-1}}{2k-2 \choose k-1}z^k \]
\[ \sum_{k=0}^{r}{r-k \choose m}{s+k \choose n} = {r+s+1 \choose m+n+1} \]
\[ C_{n, m} = {n+m \choose m} - {n+m \choose m-1}, n \geq m \]
\[ {n \choose k} \equiv [n\& k=k] \pmod 2 \]
\subsubsection{Fibonacci Numbers}
\[ F(z) = \frac{z}{1-z-z^2} \]
\[ f_n = \frac{{\phi}^n-{\hat{\phi}}^n}{\sqrt{5}}, \phi = \frac{1+\sqrt{5}}{2},
\hat{\phi} = \frac{1-\sqrt{5}}{2} \]
\[ \sum_{k=1}^nf_k = f_{n+2}-1 \]
\[ \sum_{k=1}^nf^2_k = f_nf_{n+1} \]
\[ \sum_{k=0}^nf_kf_{n-k} = \frac{1}{5}(n-1)f_n+\frac{2}{5}nf_{n-1} \]
\[ f^2_n + (-1)^n = f_{n+1}f_{n-1} \]
\[ f_{n+k} = f_nf_{k+1} + f_{n-1}f_k \]
\[ f_{2n+1} = f^2_n+f^2_{n+1} \]
\[ (-1)^kf_{n-k} = f_{n}f_{k-1} - f_{n-1}f_{k} \]
\[ \text{Modulo }f_n, f_{mn+r} \equiv \left\{
\begin{aligned}
&f_r,& m \bmod 4 = 0; \\
&(-1)^{r+1}f_{n-r},& m \bmod 4 = 1; \\
&(-1)^nf_r,& m \bmod 4 = 2; \\
&(-1)^{r+1+n}f_{n-r},& m \bmod 4 = 3.
\end{aligned}
\right.
\]
\subsubsection{Stirling Cycle Numbers}
\begin{align*}
 {n+1 \brack k} = n{n \brack k} + {n \brack k-1},&\ \  {n+1 \brack 2} = n!H_n \\
x^{\underline{n}} = \sum_k{ {n \brack k}(-1)^{n-k}x^k },&\ \  x^{\overline{n}} = \sum_k{ {n \brack k}x^k } \\
 \end{align*}
\subsubsection{Stirling Subset Numbers}
\[ {n+1 \brace k} = k{n \brace k} + {n \brace k-1} \]
\[ x^n = \sum_k{ {n \brace k}x^{\underline{k}} } = \sum_k{ {n \brace k}(-1)^{n-k}x^{\overline{k}} } \]
\[ m!{n \brace m} = \sum_k{m \choose k}k^n(-1)^{m-k} \]
\subsubsection{Eulerian Numbers}
\[ {n \bangle k} = (k+1){n-1 \bangle k} + (n-k){n-1 \bangle k-1} \]
\[ x^n = \sum_k{ {n \bangle k}{x+k \choose n} } \]
\[ {n \bangle m} = \sum_{k=0}^m{n+1 \choose k}(m+1-k)^n(-1)^k \]
\subsubsection{Harmonic Numbers}
\[ \sum_{k=1}^nH_k = (n+1)H_n-n \]
\[ \sum_{k=1}^nkH_k = \frac{n(n+1)}{2}H_n - \frac{n(n-1)}{4} \]
\[ \sum_{k=1}^n{k \choose m}H_k = {n+1 \choose m+1}(H_{n+1} - \frac{1}{m+1}) \]
\subsubsection{Pentagonal Number Theorem}
\[ \prod_{n=1}^{\infty}(1-x^n) = \sum_{n=-\infty}^{\infty}{(-1)^kx^{k(3k-1)/2}} \]
\[ p(n) = p(n-1)+p(n-2)-p(n-5)-p(n-7)+\cdots \]
\[ f(n, k) = p(n)-p(n-k)-p(n-2k)+p(n-5k)+p(n-7k)-\cdots \]
\subsubsection{Bell Numbers}
\[ B_{n+1} = \sum_{k=0}^n{n \choose k}B_k \]
\[ B_{p^m+n} \equiv mB_n+B_{n+1} \pmod{p} \]
\subsubsection{Bernoulli Numbers}
\[ B_n = 1 - \sum_{k=0}^{n-1}{n \choose k}\frac{B_k}{n-k+1} \]
\[ G(x) = \sum_{k=0}^{\infty}\frac{B_k}{k!}x^k
= \frac{1}{\sum_{k=0}^{\infty}\frac{x^k}{(k+1)!}} \]
\[ S_m(n) = \frac{1}{m+1}\sum_{k=0}^m{m+1 \choose k}B_kn^{m-k+1} \]
\subsubsection{Tetrahedron Volume}
\[ V = \frac{\sqrt{ 4u^2v^2w^2 - \sum_{cyc}{u^2(v^2+w^2-U^2)^2} + \prod_{cyc}{(v^2+w^2-U^2)} }}{12} \]
\subsubsection{BEST Thoerem}
Counting the number of different Eulerian circuits in directed graphs.
\[ \mathrm{ec}(G) = t_w(G)\prod_{v \in{V}}(\mathrm{deg}(v) - 1)! \]
When calculating $t_w(G)$ for directed multigraphs, the entry $q_{i,j}$ for distinct $i$ and $j$ equals $−m$, where $m$ is the number of edges from $i$ to $j$, and the entry $q_{i,i}$ equals the indegree of $i$ minus the number of loops at $i$.
It is a property of Eulerian graphs that $ \mathrm{tv}(G) = \mathrm{tw}(G)$ for every two vertices $v$ and $w$ in a connected Eulerian graph $G$.

\subsubsection{重心}
半径为 $r$ , 圆心角为 $\theta$ 的扇形重心与圆心的距离为 $\frac{4r\sin(\theta/2)}{3\theta}$ \\
半径为 $r$ , 圆心角为 $\theta$ 的圆弧重心与圆心的距离为 $\frac{4r\sin^3(\theta/2)}{3(\theta-\sin(\theta))}$ \\

\subsubsection{Others}
\[ S_j = \sum_{k=1}^nx_k^j \]
\[ h_m = \sum_{1\leq j_1 < \cdots < j_m \leq n} x_{j_1}\cdots x_{j_m} \]
\[ H_m = \sum_{1\leq j_1 \leq \cdots \leq j_m \leq n} x_{j_1}\cdots x_{j_m} \]
\[ h_n = \frac{1}{n}\sum_{k=1}^n(-1)^{k+1}S_kh_{n-k} \]
\[ H_n = \frac{1}{n}\sum_{k=1}^nS_kH_{n-k} \]
\[ \sum_{k=0}^nkc^k = \frac{nc^{n+2}-(n+1)c^{n+1}+c}{(c-1)^2} \]
\[ n! = \sqrt{2\pi n}(\frac{n}{e})^n(1+\frac{1}{12n}+\frac{1}{288n^2}+O(\frac{1}{n^3})) \]
\[ \begin{aligned}
 &\max{\{x_a-x_b, y_a-y_b, z_a-z_b\}} - \min{\{x_a-x_b, y_a-y_b, z_a-z_b\}} \\
=& \frac{1}{2}\sum_{cyc}\left| (x_a-y_a)-(x_b-y_b) \right|
\end{aligned} \]
\[ (a+b)(b+c)(c+a) = \frac{(a+b+c)^3 - a^3 - b^3 - c^3}{3} \]
\end{small}

\subsection{Integration Table}
\newcommand{\md}{\mathrm{d}}
\newcommand{\me}{\mathrm{e}}

\begin{small}

\subsubsection{$ax^2+bx+c$($a>0$)}

\begin{enumerate}

\item $ \int \frac{\md x}{ax^2+bx+c} = \begin{cases}
\frac{2}{\sqrt{4ac-b^2}}\arctan\frac{2ax+b}{\sqrt{4ac-b^2}} + C & (b^2 < 4ac) \\
\frac{1}{\sqrt{b^2-4ac}}\ln\left| \frac{2ax+b-\sqrt{b^2-4ac}}{2ax+b+\sqrt{b^2-4ac}} \right| + C & (b^2 > 4ac)
\end{cases} $

\item $ \int \frac{x}{ax^2+bx+c} \md x = \frac{1}{2a} \ln |ax^2+bx+c| - \frac{b}{2a} \int \frac{\md x}{ax^2+bx+c} $

\end{enumerate}

\subsubsection{$\sqrt{\pm ax^2+bx+c}$($a>0$)}

\begin{enumerate}

\item $ \int \frac{\md x}{\sqrt{ax^2+bx+c}} = \frac{1}{\sqrt{a}} \ln | 2ax+b+2\sqrt{a}\sqrt{ax^2+bx+c} | + C $

\item $ \int \sqrt{ax^2+bx+c} \md x = \frac{2ax+b}{4a}\sqrt{ax^2+bx+c} +
	\frac{4ac-b^2}{8\sqrt{a^3}}\ln |2ax+b+2\sqrt{a}\sqrt{ax^2+bx+c}| + C $

\item $ \int \frac{x}{\sqrt{ax^2+bx+c}} \md x = \frac{1}{a}\sqrt{ax^2+bx+c} -
	\frac{b}{2\sqrt{a^3}}\ln | 2ax+b+2\sqrt{a}\sqrt{ax^2+bx+c} | + C $

\item $ \int \frac{\md x}{\sqrt{c+bx-ax^2}} = -\frac{1}{\sqrt{a}} \arcsin \frac{2ax-b}{\sqrt{b^2+4ac}} + C  $

\item $ \int \sqrt{c+bx-ax^2} \md x = \frac{2ax-b}{4a}\sqrt{c+bx-ax^2} + \\
	\frac{b^2+4ac}{8\sqrt{a^3}}\arcsin\frac{2ax-b}{\sqrt{b^2+4ac}} + C $

\item $ \int \frac{x}{\sqrt{c+bx-ax^2}} \md x = -\frac{1}{a}\sqrt{c+bx-ax^2} + \frac{b}{2\sqrt{a^3}}\arcsin\frac{2ax-b}{\sqrt{b^2+4ac}} + C $

\end{enumerate}

\subsubsection{$\sqrt{\pm\frac{x-a}{x-b}}$或$\sqrt{(x-a)(x-b)}$}

\begin{enumerate}

\item $ \int \frac{\md x}{\sqrt{(x-a)(b-x)}} = 2\arcsin\sqrt\frac{x-a}{b-x} + C$ ($a<b$)

\item \begin{multline}
\int \sqrt{(x-a)(b-x)} \md x = \frac{2x-a-b}{4}\sqrt{(x-a)(b-x)} + \\
	\frac{(b-a)^2}{4}\arcsin\sqrt\frac{x-a}{b-x} + C, (a<b)
\end{multline}

\end{enumerate}

\subsubsection{三角函数的积分}

\begin{enumerate}

\item $ \int \tan x \md x = -\ln|\cos x| + C $

\item $ \int \cot x \md x = \ln |\sin x| + C $

\item $ \int \sec x \md x = \ln \left| \tan\left( \frac{\pi}{4} + \frac{x}{2} \right) \right| + C = \ln |\sec x + \tan x| + C $

\item $ \int \csc x \md x = \ln \left| \tan\frac{x}{2} \right| + C = \ln |\csc x - \cot x| + C $

\item $ \int \sec^2 x \md x = \tan x + C $

\item $ \int \csc^2 x \md x = -\cot x + C $

\item $ \int \sec x \tan x \md x = \sec x + C $

\item $ \int \csc x \cot x \md x = -\csc x + C $

\item $ \int \sin^2 x \md x = \frac{x}{2} - \frac{1}{4} \sin 2x + C $

\item $ \int \cos^2 x \md x = \frac{x}{2} + \frac{1}{4} \sin 2x + C $

\item $ \int \sin^n x \md x = -\frac{1}{n} \sin^{n-1} x \cos x + \frac{n-1}{n} \int \sin^{n-2} x \md x $

\item $ \int \cos^n x \md x = \frac{1}{n} \cos^{n-1} x \sin x + \frac{n-1}{n} \int \cos^{n-2} x \md x $

\item $ \int \frac{\md x}{\sin^n x} = -\frac{1}{n-1} \frac{\cos x}{\sin^{n-1}x} + \frac{n-2}{n-1} \int \frac{\md x}{\sin^{n-2}x} $

\item $ \int \frac{\md x}{\cos^n x} = \frac{1}{n-1} \frac{\sin x}{\cos^{n-1}x} + \frac{n-2}{n-1} \int \frac{\md x}{\cos^{n-2}x} $

\item \[ \begin{split} {} & \int \cos^m x \sin^n x \md x \\
	= & \frac{1}{m+n} \cos^{m-1} x \sin^{n+1}x + \frac{m-1}{m+n}\int\cos^{m-2}x\sin^nx\md x \\
	= & -\frac{1}{m+n} \cos^{m+1} x \sin^{n-1}x + \frac{n-1}{m+1} \int \cos^m x\sin^{n-2} x \md x \end{split} \]

\item $ \int \frac{\md x}{a + b \sin x} = \begin{cases}
\frac{2}{\sqrt{a^2-b^2}}\arctan\frac{a\tan\frac{x}{2}+b}{\sqrt{a^2-b^2}} + C & (a^2 > b^2) \\
\frac{1}{\sqrt{b^2-a^2}}\ln \left| \frac{a\tan\frac{x}{2}+b-\sqrt{b^2-a^2}}{a\tan\frac{x}{2}+b+\sqrt{b^2-a^2}} \right| + C & (a^2 < b^2)
\end{cases} $

\item $ \int \frac{\md x}{a + b \cos x} = \begin{cases}
\frac{2}{a+b}\sqrt\frac{a+b}{a-b} \arctan\left(\sqrt\frac{a-b}{a+b}\tan\frac{x}{2}\right) + C & (a^2 > b^2) \\
\frac{1}{a+b}\sqrt\frac{a+b}{a-b} \ln \left| \frac{\tan\frac{x}{2}+\sqrt\frac{a+b}{b-a}}{\tan\frac{x}{2}-\sqrt\frac{a+b}{b-a}} \right| + C
& (a^2 < b^2)
\end{cases} $

\item $ \int \frac{\md x}{a^2\cos^2x+b^2\sin^2x} = \frac{1}{ab} \arctan\left( \frac{b}{a}\tan x \right) + C $

\item $ \int \frac{\md x}{a^2\cos^2x-b^2\sin^2x} = \frac{1}{2ab}\ln\left|\frac{b\tan x+a}{b\tan x-a}\right| + C $

\item $ \int x \sin ax \md x = \frac{1}{a^2} \sin ax - \frac{1}{a} x \cos ax + C $

\item $ \int x^2 \sin ax \md x = -\frac{1}{a} x^2 \cos ax + \frac{2}{a^2} x \sin ax + \frac{2}{a^3} \cos ax + C$

\item $ \int x \cos ax \md x = \frac{1}{a^2} \cos ax + \frac{1}{a} x \sin ax + C $

\item $ \int x^2 \cos ax \md x = \frac{1}{a} x^2 \sin ax + \frac{2}{a^2} x \cos ax - \frac{2}{a^3} \sin ax + C $

\end{enumerate}

\subsubsection{反三角函数的积分(其中 $a>0$ )}

\begin {enumerate}

\item $ \int \arcsin \frac{x}{a} \md x = x \arcsin \frac{x}{a} + \sqrt{a^2-x^2}+C $

\item $ \int x \arcsin \frac{x}{a} \md x= (\frac{x^2}{2}-\frac{a^2}{4})\arcsin \frac{x}{a} + \frac{x}{4} \sqrt{x^2-x^2}+C$

\item $ \int x^2 \arcsin \frac{x}{a} \md x = \frac{x^3}{3}\arcsin \frac{x}{a}+\frac{1}{9}(x^2+2 a^2)\sqrt{a^2-x^2}+C $

\item $ \int \arccos \frac{x}{a} \md x= x \ arccos \frac{x}{a} - \sqrt{a^2-x^2} +C $

\item $ \int x \arccos \frac{x}{a} \md x= (\frac{x^2}{2}-\frac{a^2}{4})\arccos \frac{x}{a} - \frac{x}{4} \sqrt{a^2-x^2}+C $

\item $ \int x^2 \arccos \frac{x}{a}\md x= \frac{x^3}{3}\arccos \frac{x}{a} - \frac{1}{9}(x^2+2a^2)\sqrt{a^2-x^2}+C$

\item $ \int \arctan \frac{x}{a} \md x=x \arctan \frac{x}{a}-\frac{a}{2}\ln (a^2+x^2)+C $

\item $ \int x\arctan \frac{x}{a} \md x = \frac{1}{2}(a^2+x^2)\arctan \frac{x}{a} -\frac{a}{2}x+C $

\item $ \int x^2 \arctan \frac{x}{a} \md x= \frac{x^3}{3} \arctan \frac{x}{a} - \frac{a}{6}x^2 + \frac{a^3}{6} \ln (a^2+x^2)+C $

\end {enumerate}

\subsubsection{指数函数的积分}

\begin{enumerate}

\item $ \int a^x \md x= \frac{1}{\ln a} a^x + C$

\item $ \int \me ^{ax}\md x=\frac{1}{a}a^{ax}+C $ 

\item $ \int x \me  ^ {ax} \md x=\frac{1}{a^2}(ax-1)a^{ax} +C $

\item $ \int x^n \me ^{ax} \md x=\frac{1}{a}x^n \me ^{ax}-\frac{n}{a} \int x^{n-1} \me ^ {ax} \md x $

\item $ \int x a^x \md x = \frac{x}{\ln a}a^x-\frac{1}{(\ln a)^2}a^x+C $

\item $ \int x^n a^x \md x= \frac{1}{\ln a}x^n a^x-\frac{n}{\ln a}\int x^{n-1}a^x \md x $

\item $ \int \me ^{ax} \sin bx \md x = \frac{1}{a^2+b^2}\me ^{ax}(a \sin bx - b \cos bx)+C $

\item $ \int \me ^{ax} \cos bx \md x = \frac{1}{a^2+b^2}\me ^{ax}(b \sin bx + a \cos bx)+C $

\item $ \int \me ^{ax} \sin ^ n bx \md x=\frac{1}{a^2+b^2 n^2}\me ^{ax} \sin ^ {n-1} bx (a \sin bx -nb \cos bx) +\frac{n(n-1)b^2}{a^2+b^2 n^2}\int \me ^{ax} \sin ^{n-2} bx \md x $

\item $ \int \me ^{ax} \cos ^ n bx \md x=\frac{1}{a^2+b^2 n^2}\me ^{ax} \cos ^ {n-1} bx (a \cos bx +nb \sin bx) +\frac{n(n-1)b^2}{a^2+b^2 n^2}\int \me ^{ax} \cos ^{n-2} bx \md x $

\end{enumerate}

\subsubsection{对数函数的积分}

\begin{enumerate}

\item $ \int \ln x \md x = x \ln x - x + C$

\item $ \int \frac{\md x}{x \ln x} =\ln \big | \ln x \big |+C $

\item $ \int x^n \ln x \md x = \frac{1}{n+1}x^{n+1}(\ln x - \frac{1}{n+1} ) +C $

\item $ \int (\ln x)^{n} \md x = x(\ln x)^ n - n \int (\ln x)^{n-1} \md x $

\item $ \int x ^ m(\ln x)^n \md x=\frac{1}{m+1}x^{m+1} (\ln x)^n - \frac{n}{m+1} \int x^m(\ln x)^{n-1}\md x $ 

\end{enumerate}

\end{small}

\end{spacing}
\end{multicols}

\end{document}
%THE SCL ENDS
