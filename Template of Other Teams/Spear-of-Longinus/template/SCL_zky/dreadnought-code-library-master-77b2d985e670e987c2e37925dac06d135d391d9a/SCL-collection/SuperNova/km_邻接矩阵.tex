\begin{lstlisting}
#include <cstdio>
#include <cstdlib>
#include <cstring>
#include <ctime>
#include <cmath>
#include <iostream>
#include <algorithm>

using namespace std;

const int oo = 1 << 30;
int ans, value[501][501], n, m, L[501], R[501], v[501];
bool bx[501], by[501];


bool find(int now){
    bx[now] = true;
    for (int i = 1; i <= m; i++)
        if (!by[i] && L[now] + R[i] == value[now][i])
        {
           by[i] = true;
           if (!v[i] || find(v[i]))
           {
              v[i] = now;
              return(true);
           }
        }
    return(false);
}

inline void km(){
    memset(L, 0, sizeof(L));
    memset(R, 0, sizeof(R));
    for (int i = 1; i <= n; i++)
        for (int j = 1; j <= m; j++)
            L[i] = max(L[i], value[i][j]);
    ans = 0;
    memset(v, 0, sizeof(v));
    for (int i = 1; i <= min(n, m); i++) 
        for (;;)
        {
            memset(bx, false, sizeof(bx));
            memset(by, false, sizeof(by));
            if (find(i)) break;
            int Min = oo;
            for (int j = 1; j <= n; j++) 
                if (bx[j]) 
                   for (int k = 1; k <= m; k++)
                       if (!by[k]) 
                          Min = min(Min, L[j] + R[k] - value[j][k]);
            for (int j = 1; j <= n; j++) if (bx[j]) L[j] -= Min;
            for (int j = 1; j <= m; j++) if (by[j]) R[j] += Min;
        }
    for (int i = 1; i <= n; i++)
        for (int j = 1; j <= m; j++)
            if (v[j] == i) ans += value[i][j];
    printf("%d\n", abs(ans));
}

int main(){
    scanf("%d%d", &n, &m);
    for (int i = 1; i <= n; i++) 
        for (int j = 1; j <= m; j++) scanf("%d", &value[i][j]), value[i][j] = -value[i][j]; 
    km();
    for (int i = 1; i <= n; i++) 
        for (int j = 1; j <= m; j++) 
            value[i][j] = -value[i][j];
    km();
}

/*hint 500 * 500 1.5s
can solve problems whose n != m 
must be complete graph, or should change some values of matrix to satisfy the condition*/
\end{lstlisting}
